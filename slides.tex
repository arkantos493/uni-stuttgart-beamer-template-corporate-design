% !TeX spellcheck = en_US
% !TeX encoding = utf8
% !TeX program = lualatex


\documentclass[8pt, t, usernames, dvipsnames, 
aspectratio=169,% for widescreen (16:9) presentations
%aspectratio=43,% for traditional (4:3) presentations
%handout%
]{beamer}


% CUSTOM INCLUDES
% language and encodings
\usepackage[english]{babel}

% TikZ
\usepackage{tikz}
\usetikzlibrary{shapes, arrows.meta, positioning, overlay-beamer-styles, decorations}

% minted
\usepackage{minted}
\setminted{
	linenos,%
	tabsize=3,%
}
\newcommand{\cppinline}[1]{\mintinline[breaklines]{c++}{#1}}

% SIunitx
\usepackage[free-standing-units]{siunitx}
\sisetup{
	per-mode=symbol,%
	binary-units=true,%
}

% misc
\usepackage{appendixnumberbeamer}
\usepackage{booktabs}
\usepackage{csquotes}
\usepackage{iftex}
\usepackage{makecell}
\usepackage{multirow}
\usepackage{todonotes}
\usepackage{xcolor}
\usepackage{xspace}

\newcommand{\cpp}[1][]{\texttt{C++#1}\xspace}
\newcommand{\includeplot}[1]{%
	\ifluatex
	\resizebox{\textwidth}{!}{\input{#1.pgf}}
	\else
	\includegraphics[width=\textwidth]{#1.pdf}
	\fi
}

\makeatletter
\newcommand\changemode[1]{%
	\gdef\beamer@currentmode{#1}}
\makeatother

\usepackage{biblatex}
\IfFileExists{bibliography.bib}{
	\addbibresource{bibliography.bib}
}{}



%%%%%%%%%%%%%%%%%%%%%%%%%%%%%%%%%% mode options %%%%%%%%%%%%%%%%%%%%%%%%%%%%%%%%

\mode<presentation>{\usetheme{uniS}}

% define a few pictures used throughout the slides
\pgfdeclareimage[height=1.1\paperheight]{background}{figures/bg.jpg}  % background picture for title slide
\pgfdeclareimage[height=1cm]{unilogo}{.logos/unilogo.pdf} % UniS logo for title slide
\pgfdeclareimage[height=1cm]{unilogow}{.logos/unilogo_w.pdf} % (white) UniS logo for final slide
% TODO: add speaker image
\pgfdeclareimage[width=2.5cm]{speaker}{figures/speaker.pdf} % speaker photo for final slide


%%%%%%%%%%%%%%%%%%%%%%%%%%%%%%%%%% misc stuff %%%%%%%%%%%%%%%%%%%%%%%%%%%%%%%%%%

% automatically display \sectionpage at beginning of every \section{}
\AtBeginSection[]{%
\begin{frame}[plain]
  \sectionpage
\end{frame}
% uncomment block in order to display toc after every sectionpage
%\usepackage{multicol}
%\begin{frame}[c, plain]
%  \begin{footnotesize}
%    \begin{center}
%      \begin{minipage}{.75\textwidth}
%        \begin{multicols}{2}
%          \tableofcontents[currentsection]
%        \end{multicols}
%      \end{minipage}
%    \end{center}
%  \end{footnotesize}
%\end{frame}
}


%%%%%%%%%%%%%%%%%%%%%%%%%%%%%%%%%%%% setup %%%%%%%%%%%%%%%%%%%%%%%%%%%%%%%%%%%%%
% TODO: customize
\title[Title footer]{Title}
\author[Author footer]{Author}
\institute{University of Stuttgart, IPVS, SGS/SC}

%%%%%%%%%%%%%%%%%%%%%%%%%%%%%%%%%%% slides %%%%%%%%%%%%%%%%%%%%%%%%%%%%%%%%%%%%%

\begin{document}
	
    % TITLE + TABLE OF CONTENTS
	\begin{frame}[plain]
		\titlepage
	\end{frame}
	\begin{frame}{Overview}
		\tableofcontents
	\end{frame}
	
	% CONTENT
	% !TeX spellcheck = en_US
% !TeX encoding = utf8
% !TeX root = ../slides.tex

\section{First Section}

\begin{frame}[c]{First Slide}
    \centering
    Hello, World! 
    \only<2>{With overlays!}
\end{frame}


\section{Second Section}

\begin{frame}[c]{Second Slide}
    content...
\end{frame}

	% TODO: add content
	
	% FINAL SLIDE + APPENDIX
	% TODO: customize
	\finalslide{Email}{Phone number}{Fax number}
	\appendix
	\IfFileExists{content/appendix.tex}{
		% !TeX encoding = utf8
% !TeX root = ../slides.tex

\setbeamertemplate{headline}{}
\setbeamertemplate{footline}{}
\changemode{handout}


\begin{frame}[c, noframenumbering]{Appendix}

\end{frame}

	}{}
	\IfFileExists{bibliography.bib}{
		\begin{frame}{References}
			\renewcommand*{\bibfont}{\tiny}
			\printbibliography
		\end{frame}
	}{}

%\section{Einführung}
%
%\begin{frame}{Corporate Design-Folien}
%  \begin{columns}[c]
%    \begin{column}{.7\textwidth}
%      \begin{itemize}
%        \item Farben, Schriften und Formate gemäß Corporate Design vordefiniert
%        \item Basiert auf der \LaTeX Beamer-Klasse
%        \item Vorlage wird in den nächsten Monaten noch weiterentweickelt
%        \item Bilder können bspw. über \texttt{\textbackslash includegraphics} eingebunden werden
%        \item \enquote{Einbauen} von Formeln wie gewohnt: $e=mc^2$ oder bspw.
%          \begin{equation*}
%            \sum_{k=0}^{n-1} ar^k=a\frac{1-r^n}{1-r}
%          \end{equation*}
%        \item Beamer-Blöcke können verwendet werden, sind aber \emph{nicht} Corporate Design-konform
%        \item Einige Hinweise auf weitere Funktionialität im Quelltext von \texttt{example.tex}, bspw.
%          Umschaltung zwischen 4:3 und 16:9-Format
%        \item Es müssen die Schriftarten \alert{UniversforUniS65Bd-Regular} und
%          \alert{UniversforUniS45LtObl-Rg} installiert sein (erhältlich über ILIAS)
%        \item Übersetzen mit \texttt{xelatex}, ggfs. zweimal
%      \end{itemize}
%    \end{column}
%    \begin{column}{.3\textwidth}
%      \includegraphics[width=\textwidth]{fig/pdp7.jpg}
%      
%    \end{column}
%  \end{columns}
%
%
%\photocredit{DEC PDP-7 Promo-Material}
%\end{frame}

\end{document}
